\section{/home/aching/Avery/Work.School/Research/pvfs2-lock/s3asim/src/worker.h File Reference}
\label{worker_8h}\index{/home/aching/Avery/Work.School/Research/pvfs2-lock/s3asim/src/worker.h@{/home/aching/Avery/Work.School/Research/pvfs2-lock/s3asim/src/worker.h}}
{\tt \#include \char`\"{}test.h\char`\"{}}\par
\subsection*{Functions}
\begin{CompactItemize}
\item 
int \bf{worker} (int myid, int numprocs, struct \bf{mpe\_\-events\_\-s} $\ast$mpe\_\-events\_\-p, struct \bf{test\_\-params\_\-s} $\ast$test\_\-params\_\-p)
\end{CompactItemize}


\subsection{Function Documentation}
\index{worker.h@{worker.h}!worker@{worker}}
\index{worker@{worker}!worker.h@{worker.h}}
\subsubsection{\setlength{\rightskip}{0pt plus 5cm}int worker (int {\em myid}, int {\em numprocs}, struct \bf{mpe\_\-events\_\-s} $\ast$ {\em mpe\_\-events\_\-p}, struct \bf{test\_\-params\_\-s} $\ast$ {\em test\_\-params\_\-p})}\label{worker_8h_015e5ab7dd7d165f8b2d73d73e3a1a41}


This handles the high-level functions of the worker process during the majority of the application run. There is a setup phase, followed by a work phase, and then cleanup operations.

\begin{Desc}
\item[Parameters:]
\begin{description}
\item[{\em myid}]MPI myid. \item[{\em numprocs}]Number of processes used. \item[{\em mpe\_\-events\_\-p}]Pointer to timing structure. \item[{\em test\_\-params\_\-p}]Pointer to test\_\-params. \end{description}
\end{Desc}
\begin{Desc}
\item[Returns:]0 on success. \end{Desc}
